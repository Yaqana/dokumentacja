%
% Niniejszy plik stanowi przykład formatowania pracy magisterskiej na
% Wydziale MIM UW.  Szkielet użytych poleceń można wykorzystywać do
% woli, np. formatujac wlasna prace.
%
% Zawartosc merytoryczna stanowi oryginalnosiagniecie
% naukowosciowe Marcina Wolinskiego.  Wszelkie prawa zastrzeżone.
%
% Copyright (c) 2001 by Marcin Woliński <M.Wolinski@gust.org.pl>
% Poprawki spowodowane zmianami przepisów - Marcin Szczuka, 1.10.2004
% Poprawki spowodowane zmianami przepisow i ujednolicenie 
% - Seweryn Karłowicz, 05.05.2006
% Dodanie wielu autorów i tłumaczenia na angielski - Kuba Pochrybniak, 29.11.2016

% dodaj opcję [licencjacka] dla pracy licencjackiej
% dodaj opcję [en] dla wersji angielskiej (mogą być obie: [licencjacka,en])
\documentclass[licencjacka]{style}


% Dane magistranta:
%\autor{Imię Nazwisko}{123456}


% Dane magistrantów:
\autor{Magdalena Grabowska}{372701}
\autori{Michał Kukuła}{371127}
\autorii{Klaudia Laks}{371151}
\autoriii{Przemysław Perkowski}{371308}
%\autoriv{Autor nr Cztery}{432145}
%\autorv{Autor nr Pięć}{342011}

\title{Miejsca Obsługi Podróżnych}


\tytulang{Parking places at rest areas on highways and expressways in Poland}

%kierunek: 
% - matematyka, informacyka, ...
% - Mathematics, Computer Science, ...
\kierunek{informatyka}

% informatyka - nie okreslamy zakresu (opcja zakomentowana)
% matematyka - zakres moze pozostac nieokreslony,
% a jesli ma byc okreslony dla pracy mgr,
% to przyjmuje jedna z wartosci:
% {metod matematycznych w finansach}
% {metod matematycznych w ubezpieczeniach}
% {matematyki stosowanej}
% {nauczania matematyki}
% Dla pracy licencjackiej mamy natomiast
% mozliwosc wpisania takiej wartosci zakresu:
% {Jednoczesnych Studiow Ekonomiczno--Matematycznych}

% \zakres{Tu wpisac, jesli trzeba, jedna z opcji podanych wyzej}

% Praca wykonana pod kierunkiem:
% (podać tytuł/stopień imię i nazwisko opiekuna
% Instytut
% ew. Wydział ew. Uczelnia (jeżeli nie MIM UW))
\opiekun{dr. hab. Aleksego Schuberta\\
  }

% miesiąc i~rok:
\date{Czerwiec 2018}

%Podać dziedzinę wg klasyfikacji Socrates-Erasmus:
\dziedzina{ 
%11.0 Matematyka, Informatyka:\\ 
%11.1 Matematyka\\ 
%11.2 Statystyka\\ 
11.3 Informatyka\\ 
%11.4 Sztuczna inteligencja\\ 
%11.5 Nauki aktuarialne\\
%11.9 Inne nauki matematyczne i informatyczne
}

%TODO Tu trzeba wpisać jakieś numerki ale nie umiem ich znaleźć.
%Klasyfikacja tematyczna wedlug AMS (matematyka) lub ACM (informatyka)
\klasyfikacja{D. Software \\ \\}


%TODO napisać jakieś
%Słowa kluczowe:
\keywords{symulacja, wizualizacja, Miejsce Obsługi Podróżnych, miejsce parkingowe, aplikacja okienkowa, aplikacja mobilna, strona internetowa, zajętość miejsc parkingowych, natężenie ruchu}

% Tu jest dobre miejsce na Twoje własne makra i~środowiska:
\newtheorem{defi}{Definicja}[section]

% koniec definicji

\begin{document}

\maketitle

%tu idzie streszczenie na strone poczatkowa
\begin{abstract}
  W~pracy opisano implementację systemu dotyczącego Miejsc Obsługi Podróżnych
  przy autostradach i~drogach ekspresowych w Polsce. Podstawowe składowe tego
  systemu to aplikacje Mopnik i~Mopsim. Są one aplikacjami okienkowymi
  korzystającymi ze wspólnego interfejsu graficznego. Służą do~przeprowadzania
  krótko- i~długoterminowych predykcji ruchu na drogach oraz zajętości miejsc
  parkingowych na~Miejscach Obsługi Podróżnych. Pozostałe dwie części to
  aplikacja mobilna oraz strona internetowa przeznaczone dla kierowców
  poruszających się po drogach. Informują one o~zajętości miejsc parkingowych
  na każdym MOPie w danym momencie oraz predyckję ich zajętości w niedalekiej
  przyszłości.  
  %TODO napisać coś jak będzie co streszczać 
\end{abstract}

\tableofcontents
%\listoffigures
%\listoftables

\chapter*{Wprowadzenie}
\addcontentsline{toc}{chapter}{Wprowadzenie}

Miejsca Obsługi Podróżnych to wydzielone obszary znajdujące się w pobliżu dróg.
Są wyposażone między innymi w~miejsca parkingowe. Podstawowe zagadnienia
dotyczące budowania nowych Miejsc Obsługi Podróżnych to ich lokalizacja oraz
liczba miejsc parkingowych. Powinny odpowiadać aktualnym potrzebom to znaczy być
uzależnione od~czynników takich jak gęstość ruchu na~danym odcinku, odległości od~istniejących
Miejsc Obsługi Podróżnych, czy węzłów komunikacyjnych. 

System służący do wykonywania takich analiz powinien również dawać
możliwość wykonania symulacji z uwzględnieniem prognozowanych zmian w ruchu
drogowym, w tym budowy nowych dróg.

Obecnie w~Polsce oraz w kilku innych europejskich krajach takie symulacje przeprowadzane są za~pomocą programu
MATSim\footnote{}. %TODO dopisać jakiś opis tego programu 
%Napisać czemu go zmieniamy


\chapter{Podstawowe pojęcia}\label{r:pojecia}

\section{Definicje}

\begin{defi}\label{GDDKiA}
  \emph{GDDKiA} - Generalna Dyrekcja Dróg Krajowych i Autostrad.
\end{defi}

\begin{defi}\label{MOP}
  \emph{MOP} - Miejsce Obsługi Podróżnych.
\end{defi}

\begin{defi}\label{SDR}
  \emph{SDR} - Średniodobowe natężenie ruchu.
\end{defi}

\begin{defi}\label{GUI}
  \emph{GUI} - Graficzny interfejs użytkownika.
\end{defi}

\begin{defi}\label{OSM}
  \emph{OSM} - Serwis OpenStreetMap (http://openstreetmap.org).
\end{defi}

\section{Blabalizator różnicowy}

Teoretycy blabalii (zob. np. pracę~\cite{grglo}) zadowalają się
niekonstruktywnym opisem natury fetorów.

Podstawowym narzędziem blabalii empirycznej jest blabalizator
różnicowy.  Przyrząd ten pozwala w~sposób przybliżony uzyskać
współczynniki rozkładu Głombaskiego dla fetorów bazowych
i~harmonicznych.  Praktyczne znaczenie tego procesu jest oczywiste:
korzystając z~reperkusatywności pozwala on przejść do przestrzeni
$\Lambda^{\nabla}$, a~tym samym znaleźć retroizotonalne współczynniki
semi-quasi-celibatu dla klatek Rozkoszy (zob.~\cite{JR}).

Klasyczne algorytmy dla blabalizatora różnicowego wykorzystują:
\begin{enumerate}
\item dualizm falowo-korpuskularny, a w szczególności
  \begin{enumerate}
  \item korpuskularną naturę fetorów,
  \item falową naturę blaba,
  \item falowo-korpuskularną naturę gryzmołów;
  \end{enumerate}
\item postępującą gryzmolizację poszczególnych dziedzin nauki, w
  szczególności badań systemowych i rozcieńczonych;
\item dynamizm fazowy stetryczenia parajonizacyjnego;
\item wreszcie tradycyjne opozycje:
  \begin{itemize}
  \item duch --- bakteria,
  \item mieć --- chcieć,
  \item myśl --- owsianka,
  \item parafina --- durszlak\footnote{Więcej o tym przypadku --- patrz
      prace Gryzybór-Głombaskiego i innych teoretyków nurtu
      teoretyczno-praktycznego badań w~Instytucie Podstawowych
      Problemów Blabalii w~Fifie.},
  \item logos --- termos%\footnote{Szpotański}
  \end{itemize}
  z właściwym im przedziwym dynamizmem.
\end{enumerate}

\begin{figure}[tp]
  \centering
  \framebox{\vbox to 4cm{\vfil\hbox to
      7cm{\hfil\tiny.\hfil}\vfil}}
  \caption{Artystyczna wizja blaba w~obrazie węgierskiego artysty
    Josipa~A. Rozkoszy pt.~,,Blaba''}
\end{figure}

\chapter{Mopnik}\label{r:mopnik}

\section{Podejście wprost}

Najprostszym sposobem wykonania blabalizy jest siłowe przeszukanie
całej przestrzeni rozwiązań.  Jednak, jak łatwo wyliczyć, rozmiar
przestrzeni rozwiązań rośnie wykładniczo z~liczbą fetorów bazowych.
Tak więc przegląd wszystkich rozwiązań sprawdza się jedynie dla bardzo
prostych przestrzeni lamblialnych.  Oznacza to, że taka metoda ma
niewielkie znaczenie praktyczne --- w~typowym przypadku z~życia trzeba
rozważać przestrzenie lamblialne wymiaru rzędu 1000.

W~literaturze można znaleźć kilka prób opracowania heurystyk dla
problemu blabalizy (por. \cite{heu}).  Korzystając z~heurystyk daje
się z~pewnym trudem dokonać blabalizy w~przestrzeni o~np.~500 fetorach
bazowych.  Należy jednak pamiętać, że heurystyka nie daje gwarancji
znalezienia najlepszego rozwiązania.  Fifak w~pracy~\cite{ff-sr}
podaje, jak dla dowolnie zadanej funkcji oceniającej skonstruować
dane, dla których rozwiązanie wygenerowane przez algorytm heurystyczny
jest dowolnie odległe od rzeczywistego.

\section{Metody wykorzystujące teorię Głombaskiego}

Teoria Głombaskiego (zob.~\cite{grglo}) dostarcza eleganckiego
narzędzia opisu przejścia do przestrzeni $\Lambda^{\nabla}$.
Wystarczy mianowicie przedstawić fetory bazowe wyjściowej przestrzeni
lamblialnej w~nieskończonej bazie tak zwanych wyższych aromatów.
(Formalną definicję tego pojęcia przedstawię w~rozdziale poświęconym
teorii Fifaka).  Podstawową cechą wyższych aromatów jest ulotność.  To
zaś oznacza, że odpowiednio dobierając współczynniki przejścia do
przestrzeni wyższych aromatów można zagwarantować dowolną z~góry
zadaną dokładność przybliżonego rozwiązania problemu blabalizy.

Oczywiście ze względu na nieskończony wymiar przestrzeni wyższych
aromatów koszt poszukiwania współczynników blabalizy jest liniowy ze
względu na wymiar wyjściowej przestrzeni lamblialnej.

\section{Metody wykorzystujące własności fetorów $\sigma$}

Najchętniej wykorzystywaną przestrzenią wyższych aromatów jest
przestrzeń fetorów~$\sigma$.  Fetory $\sigma$ dają szczególnie prostą
bazę podprzestrzeni widłowej.  Wiąże się to z~faktem, że w~tym przypadku
fetory harmoniczne wyższych rzędów są pomijalne (rzędu $2^{-t^3}$,
gdzie $t$ jest wymiarem wyjściowej przestrzeni lamblialnej).

Niestety z~fetorami $\sigma$ wiąże się też przykre ograniczenie: można
wykazać (zob.~\cite[s. 374]{ff-sr}), że dla dowolnie dobranej bazy
w~podprzestrzeni widłowej istnieje ograniczenie dolne w~metryce sierpa
na odległość rzutu dokładnego rozwiązania problemu blabalizy na
podprzestrzeń widłową.  Ponieważ rzut ten stanowi najlepsze
przybliżone rozwiązanie, jakie można osiągnąć nie naruszając aksjomatu
reperkusatywności, więc istnieje pewien nieprzekraczalny próg
dokładności dla blabalizy wykonanej przez przejście do przestrzeni
fetorów $\sigma$.  Wartość retroinicjalną tego progu nazywa się
\textit{reziduum blabicznym}.

\chapter{Mopsim}\label{r:mopsim}

Głównym odkryciem Fifaka jest, że fetor suprakowariantny może
gryzmolizować dowolny ideał w~podprzestrzeni widłowej przestrzeni
lamblialnej funkcji Rozkoszy.

Udowodnienie tego faktu wymagało wykorzystania twierdzeń pochodzących
z~kilku niezależnych teorii matematycznych (zob. na przykład:
\cite{russell,spyrpt,JR,beaman,hopp,srinis}).  Jednym z~filarów
dowodu jest teoria odwzorowań owalnych Leukocyta (zob.~\cite{leuk}).

Znaczenie twierdzenia Fifaka dla problemu blabalizy polega na tym, że
znając retroizotonalne współczynniki dla klatek Rozkoszy można
przeprowadzić fetory bazowe na dwie nieskończone bazy fetorów $\sigma$
w~przestrzeni $K_7$ i~fetorów $\rho$ w~odpowiedniej
quasi-quasi-przestrzeni równoległej (zob.~\cite{hopp}).  Zasadnicza
różnica w~stosunku do innych metod blabalizy polega na tym, że
przedstawienie to jest dokładne.

\chapter{Dokumentacja użytkowa i~opis implementacji}\label{r:impl}

Program przygotowany dla systemu operacyjnego M\$ Windows uruchamia
się energicznym dwumlaskiem na jego ikonce w~folderze
\verb+\\FIDO\FOO\BLABA+.  Następnie kolistym ruchem ręki należy
naprowadzić kursor na menu \texttt{Blabaliza} i~uaktywnić pozycję
\texttt{Otwórz plik}.  Po wybraniu pliku i~zatwierdzeniu wyboru
przyciskiem \texttt{OK} rozpocznie się proces blabalizy.  Wyniki
zostaną zapisane w~pliku o~nazwie \texttt{99-1a.tx.43} w~bieżącym
folderze.

\chapter{Aplikacja mobilna i strona internetowa}\label{r:apka} 

\chapter{Podsumowanie}

W~pracy przedstawiono pierwszą efektywną implementację blabalizatora
różnicowego.  Umiejętność wykonania blabalizy numerycznej dla danych
,,z życia'' stanowi dla blabalii fetorycznej podobny przełom, jak dla
innych dziedzin wiedzy stanowiło ogłoszenie teorii Mikołaja Kopernika
i~Gryzybór Głombaskiego.  Z~pewnością w~rozpocznynającym się XXI wieku
będziemy obserwować rozkwit blabalii fetorycznej.

Trudno przewidzieć wszystkie nowe możliwości, ale te co bardziej
oczywiste można wskazać już teraz.  Są to:
\begin{itemize}
\item degryzmolizacja wieńców telecentrycznych,
\item realizacja zimnej reakcji lambliarnej,
\item loty celulityczne,
\item dokładne obliczenie wieku Wszechświata.
\end{itemize}

\section{Perspektywy wykorzystania w~przemyśle}

Ze względu na znaczenie strategiczne wyników pracy ten punkt uległ
utajnieniu.

\appendix

\chapter{Główna pętla programu zapisana w~języku T\=oFoo}

\begin{verbatim}
[[foo]{,}[[a3,(([(,),{[[]]}]),
  [1; [{,13},[[[11],11],231]]].
  [13;[!xz]].
  [42;[{,x},[[2],{'a'},14]]].
  [br;[XQ*10]].
 ), 2q, for, [1,]2, [..].[7]{x}],[(((,[[1{{123,},},;.112]],
        else 42;
   . 'b'.. '9', [[13141],{13414}], 11),
 [1; [[134,sigma],22]].
 [2; [[rho,-],11]].
 )[14].
 ), {1234}],]. [map [cc], 1, 22]. [rho x 1]. {22; [22]},
       dd.
 [11; sigma].
        ss.4.c.q.42.b.ll.ls.chmod.aux.rm.foo;
 [112.34; rho];
        001110101010101010101010101010101111101001@
 [22%f4].
 cq. rep. else 7;
 ]. hlt
\end{verbatim}

\chapter{Przykładowe dane wejściowe algorytmu}

\begin{center}
  \begin{tabular}{rrr}
    $\alpha$ & $\beta$ & $\gamma_7$ \\
    901384 & 13784 & 1341\\
    68746546 & 13498& 09165\\
    918324719& 1789 & 1310 \\
    9089 & 91032874& 1873 \\
    1 & 9187 & 19032874193 \\
    90143 & 01938 & 0193284 \\
    309132 & $-1349$ & $-149089088$ \\
    0202122 & 1234132 & 918324098 \\
    11234 & $-109234$ & 1934 \\
  \end{tabular}
\end{center}

\chapter{Przykładowe wyniki blabalizy
    (ze~współczynnikami~$\sigma$-$\rho$)}

\begin{center}
  \begin{tabular}{lrrrr}
    & Współczynniki \\
    & Głombaskiego & $\rho$ & $\sigma$ & $\sigma$-$\rho$\\
    $\gamma_{0}$ & 1,331 & 2,01 & 13,42 & 0,01 \\
    $\gamma_{1}$ & 1,331 & 113,01 & 13,42 & 0,01 \\
    $\gamma_{2}$ & 1,332 & 0,01 & 13,42 & 0,01 \\
    $\gamma_{3}$ & 1,331 & 51,01 & 13,42 & 0,01 \\
    $\gamma_{4}$ & 1,332 & 3165,01 & 13,42 & 0,01 \\
    $\gamma_{5}$ & 1,331 & 1,01 & 13,42 & 0,01 \\
    $\gamma_{6}$ & 1,330 & 0,01 & 13,42 & 0,01 \\
    $\gamma_{7}$ & 1,331 & 16435,01 & 13,42 & 0,01 \\
    $\gamma_{8}$ & 1,332 & 865336,01 & 13,42 & 0,01 \\
    $\gamma_{9}$ & 1,331 & 34,01 & 13,42 & 0,01 \\
    $\gamma_{10}$ & 1,332 & 7891432,01 & 13,42 & 0,01 \\
    $\gamma_{11}$ & 1,331 & 8913,01 & 13,42 & 0,01 \\
    $\gamma_{12}$ & 1,331 & 13,01 & 13,42 & 0,01 \\
    $\gamma_{13}$ & 1,334 & 789,01 & 13,42 & 0,01 \\
    $\gamma_{14}$ & 1,331 & 4897453,01 & 13,42 & 0,01 \\
    $\gamma_{15}$ & 1,329 & 783591,01 & 13,42 & 0,01 \\
  \end{tabular}
\end{center}

\begin{thebibliography}{99}
\addcontentsline{toc}{chapter}{Bibliografia}

\bibitem[Bea65]{beaman} Juliusz Beaman, \textit{Morbidity of the Jolly
    function}, Mathematica Absurdica, 117 (1965) 338--9.

\bibitem[Blar16]{eb1} Elizjusz Blarbarucki, \textit{O pewnych
    aspektach pewnych aspektów}, Astrolog Polski, Zeszyt 16, Warszawa
  1916.

\bibitem[Fif00]{ffgg} Filigran Fifak, Gizbert Gryzogrzechotalski,
  \textit{O blabalii fetorycznej}, Materiały Konferencji Euroblabal
  2000.

\bibitem[Fif01]{ff-sr} Filigran Fifak, \textit{O fetorach
    $\sigma$-$\rho$}, Acta Fetorica, 2001.

\bibitem[Głomb04]{grglo} Gryzybór Głombaski, \textit{Parazytonikacja
    blabiczna fetorów --- nowa teoria wszystkiego}, Warszawa 1904.

\bibitem[Hopp96]{hopp} Claude Hopper, \textit{On some $\Pi$-hedral
    surfaces in quasi-quasi space}, Omnius University Press, 1996.

\bibitem[Leuk00]{leuk} Lechoslav Leukocyt, \textit{Oval mappings ab ovo},
  Materiały Białostockiej Konferencji Hodowców Drobiu, 2000.

\bibitem[Rozk93]{JR} Josip A.~Rozkosza, \textit{O pewnych własnościach
    pewnych funkcji}, Północnopomorski Dziennik Matematyczny 63491
  (1993).

\bibitem[Spy59]{spyrpt} Mrowclaw Spyrpt, \textit{A matrix is a matrix
    is a matrix}, Mat. Zburp., 91 (1959) 28--35.

\bibitem[Sri64]{srinis} Rajagopalachari Sriniswamiramanathan,
  \textit{Some expansions on the Flausgloten Theorem on locally
    congested lutches}, J. Math.  Soc., North Bombay, 13 (1964) 72--6.

\bibitem[Whi25]{russell} Alfred N. Whitehead, Bertrand Russell,
  \textit{Principia Mathematica}, Cambridge University Press, 1925.

\bibitem[Zen69]{heu} Zenon Zenon, \textit{Użyteczne heurystyki
    w~blabalizie}, Młody Technik, nr~11, 1969.

\end{thebibliography}

\end{document}


%%% Local Variables:
%%% mode: latex
%%% TeX-master: t
%%% coding: latin-2
%%% End:
